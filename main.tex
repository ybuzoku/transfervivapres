\documentclass{beamer}
\usetheme{ucl}

\newcommand\hmmax{0}
\newcommand\bmmax{0}

\usepackage{graphicx}
\usepackage[all]{xy}

% \usepackage[utf8]{inputenc}
\usepackage{amsmath,amsthm,stmaryrd,textcomp,amssymb,enumerate}
\usepackage{mathbbol}
\usepackage{bbold}
\usepackage{dsfont}
\usepackage{algorithm} % Boxes/formatting around algorithms
\usepackage{mathtools}
%\usepackage[noend]{algpseudocode} %
\usepackage{hyperref}
\usepackage{xcolor}
\hypersetup{
    colorlinks=true,
    urlcolor     = blue, %Colour for external hyperlinks
    linkcolor    = blue, %Colour of internal links
    citecolor    = blue %Colour of citations
}
% \usepackage{amsthm}
\usepackage{comment} % To comment paragraphs.
\usepackage{color}
\usepackage{relsize}
\usepackage{bm}
\usepackage{bbm} % For mathbb numbers.
\usepackage{subcaption} % Subfigure.
\usepackage{cleveref}
\usepackage{bussproofs}
%\usepackage{ebproof}
\usepackage{pifont}
\usepackage{wrapfig}
\usepackage{cutwin}

\usepackage{tikz-cd}
\usepackage{tikzit}
\input{local.tikzstyles}
\usepackage{mathrsfs}
%\usepackage{proof}
% NEW IMPORTS
\usepackage{cmll}       % For \with
\newenvironment{bprooftree}
  {\leavevmode\hbox\bgroup}
  {\DisplayProof\egroup}
  
%%%%%%%%%%%%%% Define the new environments here;
%\newenvironment{qparts}{\begin{enumerate}[{(}a{)}]}{\end{enumerate}}
% \def\endproofmark{$\Box$}
%\renewenvironment{proof}{\par{\bf Proof}:}{\hfill$\blacksquare$}

\newtheorem{defn}[theorem]{Definition}

\newtheorem{question}[theorem]{Question}
\newtheorem{remark}[theorem]{Remark}
\newtheorem{construction}[theorem]{Construction}
\newtheorem{observation}[theorem]{Observation}

%Make font sans-serif
\renewcommand{\familydefault}{\sfdefault}

%%%%%% Define commands here;

% Basic logic notation.
%\newcommand{\seq}{\triangleright}
\newcommand{\seq}{\rightsquigarrow }
\newcommand{\sequent}[2]{{#1} \seq {#2}}
\newcommand{\thmSequent}[2]{{#1} \vdash {#2}}

%%%%%%%%%%%%%%%%%%%%%%%%%%%%%%%%%%%%%%%%%
%       BeS
%%%%%%%%%%%%%%%%%%%%%%%%%%%%%%%%%%%%%%%%%
\newcommand{\base}[1]{\mathscr{#1}}
\newcommand{\baseB}{\base{B}}
\newcommand{\baseC}{\base{C}}
\newcommand{\baseD}{\base{D}}
\newcommand{\baseE}{\base{E}}
\newcommand{\baseF}{\base{F}}
\newcommand{\baseG}{\base{G}}
\newcommand{\baseH}{\base{H}}
\newcommand{\baseX}{\base{X}}
\newcommand{\baseY}{\base{Y}}
\newcommand{\baseZ}{\base{Z}}
\newcommand{\baseIMALL}{\base{N}}
\newcommand{\emptybase}{\varnothing}
\newcommand{\basis}[1]{\mathfrak{#1}}
% \newcommand{\at}[1]{\mathrm{#1}}
\newcommand{\at}[1]{{#1}}
\newcommand{\At}{\mathrm{At}}

%\newcommand{\baseGeq}{\succeq}
%\newcommand{\baseLeq}{\preceq}
\newcommand{\baseGeq}{\supseteq}
\newcommand{\baseLeq}{\subseteq}

\newcommand{\suppTor}[1]{\Vdash_{\!\!#1}}
% \newcommand{\supp}[1]{\Vdash_{\!\!#1}}
% \newcommand{\suppNew}[1]{\vDash_{\!#1}}
\newcommand{\suppNew}[1]{\Vdash^{*}_{\!#1}}
\newcommand{\entails}{\Vdash}
%\newcommand{\proves}[1]{\vdash_{\!\!#1}}
\newcommand{\proves}{\vdash}
\newcommand{\plus}{+}
\newcommand{\system}[1]{\mathsf{#1}}
\newcommand{\Atoms}{\set{A}}
\newcommand{\Formulas}{\set{F}}
\newcommand{\Bunches}{\set{B}}
\newcommand{\set}[1]{\mathbb{#1}}


% ILL
\newcommand{\IPL}{IPL}
\newcommand{\IMLL}{IMLL}
\newcommand{\IMALL}{IMALL}
\newcommand{\ILL}{ILL}
\newcommand{\provesIMALL}{\vdash_{\mathrm{IMALL}}}
\newcommand{\provesILL}{\vdash_{\mathrm{ILL}}}
\newcommand{\multiset}{\mathcal{M}}
\newcommand{\fmultiset}{\mathcal{M}_{\!{f}}}
\newcommand{\emptymultiset}{\varnothing}
\newcommand{\deriveBaseM}[1]{\vdash_{\!\!#1}}
\newcommand{\suppM}[2]{\Vdash_{ \!\!#1 }^{ \!\!#2 }}
\newcommand{\suppMAlt}[2]{\Vvdash_{ \!\!#1 }^{ \!\!#2 }}
\newcommand{\IMALLformula}{\mathsf{Form}}
\newcommand{\mand}{\otimes}
\newcommand{\mtop}{1}
\newcommand{\mto}{\multimap}
\newcommand{\aand}{\with}
\newcommand{\aor}{\oplus}
\newcommand{\abot}{0}
\newcommand{\msetunion}{,}
\newcommand{\bang}{\mathop{!}}
\newcommand{\makeMultiset}[1]{[#1]}
\newcommand{\flatIMALL}[1]{{#1}^{\flat}}
\newcommand{\deflatIMALL}[1]{{#1}^{\natural}}
\newcommand{\openaddrule}{\{}
\newcommand{\closeaddrule}{\}}
\newcommand{\Closeaddrule}[2]{\}^{#2}_{#1}}
\newcommand{\extAt}{\widetilde{\At}}
\DeclareMathSymbol{\fatcomma}{\mathrel}{bbold}{\lq\,}

%Misc
\newbox\qqBoxA
\newdimen\qqCornerHgt
\setbox\qqBoxA=\hbox{$\ulcorner$}
\global\qqCornerHgt=\ht\qqBoxA
\newdimen\qqArgHgt
\def\Quinequote #1{%
    \setbox\qqBoxA=\hbox{$#1$}%
    \qqArgHgt=\ht\qqBoxA%
    \ifnum     \qqArgHgt<\qqCornerHgt \qqArgHgt=0pt%
    \else \advance \qqArgHgt by -\qqCornerHgt%
    \fi \raise\qqArgHgt\hbox{$\ulcorner$} \box\qqBoxA %
    \raise\qqArgHgt\hbox{$\urcorner$}}



\setbeamersize{description width=2em}
%%% Remove nav symbols (and shift any logo down to corner)
\setbeamertemplate{navigation symbols}{\vspace{-2ex}}
\setbeamertemplate{footline}[author title date]
\setbeamercolor{banner}{bg=darkpurple}
\useinnertheme{blockborder}

%\usepackage{handoutWithNotes}
%\pgfpagesuselayout{4 on 1 with notes}[a4paper,border shrink=5mm]

\graphicspath{{./images/} }
\title[P-tS for ILL]{Transfer viva: \newline Proof-theoretic Semantics for ILL and beyond}
%\author{\texorpdfstring{Yll Buzoku\newline\url{y.buzoku@ucl.ac.uk}}{https://www.homepages.ucl.ac.uk/~zcapybu/}}
\author{Yll Buzoku}
\institute[UCL]{%
  Department of Computer Science \\ %
  University College London
}
\date{May 30, 2024}

%%%%% Uncomment the bottom 7 lines to introduce between ToC between sections
%\AtBeginSection[]
%{
%  \begin{frame}
%    \frametitle{Presentation root directory}
%    \tableofcontents[currentsection]
%  \end{frame}
%}

\begin{document}
%%%%%%%%%%%%%%%%%%%%%%%%%%%%%%%%%%%%%%%%%%%%%%%%%%%%%%%%
%%%%%%%%%%%%%%%%%%%%%%%%%%%%%%%%%%%%%%%%%%%%%%%%%%%%%%%%
\begin{frame}
\titlepage
\end{frame}
%%%%%%%%%%%%%%%%%%%%%%%%%%%%%%%%%%%%%%%%%%%%%%%%%%%%%%%%
%%%%%%%%%%%%%%%%%%%%%%%%%%%%%%%%%%%%%%%%%%%%%%%%%%%%%%%%
\section*{Goals for this presentation}
\begin{frame}{Goals for this talk}
\begin{itemize}
\item To cover the work done in obtaining a proof-theoretic semantics for ILL.
\item To highlight some interesting aspects of the results obtained.
\item To present an alternative semantics for ILL.
\item To discuss my future plans for my PhD.
\end{itemize}
\end{frame}
%%%%%%%%%%%%%%%%%%%%%%%%%%%%%%%%%%%%%%%%%%%%%%%%%%%%%%%%
%%%%%%%%%%%%%%%%%%%%%%%%%%%%%%%%%%%%%%%%%%%%%%%%%%%%%%%%
%%%%%%%%%%%%%%%%%%%%%%%%%%%%%%%%%%%%%%%%%%%%%%%%%%%%%%%%
\begin{frame}{A natural deduction system for ILL}
	\begin{center}
	\noindent\begin{minipage}{0.47\textwidth}\scriptsize
	\begin{prooftree}
	  \def\ScoreOverhang{0.5pt}
		  \AxiomC{}
		  \RightLabel{Ax}
		  \UnaryInfC{$\varphi\proves\varphi$}
	\end{prooftree}
	\end{minipage}
	\vspace{2pt}
	\noindent\begin{minipage}{0.47\textwidth}\scriptsize
		\begin{prooftree}
		  \def\ScoreOverhang{0.5pt}
			  \AxiomC{$\Gamma,\varphi \proves \psi$}
			  \RightLabel{$\mto$-I}
			  \UnaryInfC{$\Gamma \proves \varphi \mto \psi$}
		\end{prooftree}
		\begin{prooftree}
		\def\ScoreOverhang{0.5pt}
			\AxiomC{$\Gamma \proves \varphi$}
			\AxiomC{$\Delta \proves \psi$}
			\RightLabel{$\mand$-I}
			\BinaryInfC{$\Gamma,\Delta \proves \varphi\mand\psi$}
		\end{prooftree}
		\begin{prooftree}
		\def\ScoreOverhang{0.5pt}
			\AxiomC{}
			\RightLabel{$\mtop$-I}
			\UnaryInfC{$\proves \mtop$}
		\end{prooftree}
		\begin{prooftree}
		\def\ScoreOverhang{0.5pt}
			\AxiomC{$\Gamma \proves \varphi$}
			\AxiomC{$\Gamma \proves \psi$}
			\RightLabel{$\aand$-I}
			\BinaryInfC{$\Gamma \proves \varphi \aand \psi$}
		\end{prooftree}
		\begin{prooftree}
		\def\ScoreOverhang{0.5pt}
			\AxiomC{$\Gamma \proves \varphi_i$}
			\RightLabel{$\aor$-$\text{I}_i$}
			\UnaryInfC{$\Gamma \proves \varphi_0 \aor \varphi_1$}
		\end{prooftree}
		\begin{prooftree}
		\def\ScoreOverhang{0.5pt}
			\AxiomC{$\Gamma_1 \provesILL \varphi_1$}
			\AxiomC{$\dots$}
			\AxiomC{$\Gamma_n \provesILL \varphi_n$}
			\RightLabel{$\top$-I}
			\TrinaryInfC{$\Gamma_1,\dots,\Gamma_n \provesILL \top$}
		\end{prooftree}
	\end{minipage}%%%%%%%%%%%%%%%%%%%%%%%%%%%%%%%%
	\begin{minipage}{0.47\textwidth}\scriptsize
		\begin{prooftree}
		  \def\ScoreOverhang{0.5pt}
			  \AxiomC{$\Gamma \proves \varphi \mto \psi$}
			  \AxiomC{$\Delta \proves \varphi$}
			  \RightLabel{$\mto$-E}
			  \BinaryInfC{$\Gamma,\Delta \proves \psi$}
		\end{prooftree}
		\begin{prooftree}
		\def\ScoreOverhang{0.5pt}
			\AxiomC{$\Gamma \proves \varphi\mand\psi$}
			\AxiomC{$\Delta, \varphi, \psi \proves \chi$}
			\RightLabel{$\mand$-E}
			\BinaryInfC{$\Gamma, \Delta \proves \chi$}
		\end{prooftree}
		\begin{prooftree}
		\def\ScoreOverhang{0.5pt}
			\AxiomC{$\Gamma \proves \varphi$}
			\AxiomC{$\Delta \proves \mtop$}
			\RightLabel{$\mtop$-E}
			\BinaryInfC{$\Gamma,\Delta \proves \varphi$}
		\end{prooftree}
		\begin{prooftree}
		\def\ScoreOverhang{0.5pt}
			\AxiomC{$\Gamma \proves \varphi_0 \aand \varphi_1$}
			\RightLabel{$\aand$-$\text{E}_i$}
			\UnaryInfC{$\Gamma \proves \varphi_i$}
		\end{prooftree}
		\begin{prooftree}
		\def\ScoreOverhang{0.5pt}
			\AxiomC{$\Gamma \proves \varphi \aor \psi$}
			\AxiomC{$\Delta,\varphi \proves \chi$}
			\AxiomC{$\Delta,\psi \proves \chi$}
			\RightLabel{$\aor$-E}
			\TrinaryInfC{$\Gamma, \Delta \proves \chi$}
		\end{prooftree}
		\begin{prooftree}
		\def\ScoreOverhang{0.5pt}
			\AxiomC{$\Gamma \proves \abot$}
			\RightLabel{$\abot$-E}
			\UnaryInfC{$\Gamma \proves \varphi$}
		\end{prooftree}
	\end{minipage}
	\end{center}
	\end{frame}
	%%%%%%%%%%%%%%%%%%%%%%%%%%%%%%%%%%%%%%%%%%%%%%%%%%%%%%%%
	\begin{frame}{A natural deduction system for ILL (cont.)}
	\begin{center}
	\begin{minipage}{0.75\textwidth}\scriptsize
		\begin{prooftree}
			\def\ScoreOverhang{0.5pt}
				\AxiomC{$\Gamma_1 \provesILL \bang\psi_1$}
				\AxiomC{$\ldots$}
				\AxiomC{$\Gamma_n \provesILL \bang\psi_n$}
				\AxiomC{$\bang\psi_1,\ldots, \bang\psi_n \provesILL \varphi$}
				\RightLabel{$\bang$-Promotion}
				\QuaternaryInfC{$\Gamma_1, \ldots,\Gamma_n \provesILL \bang\varphi$}
		\end{prooftree}
		\begin{prooftree}
			\def\ScoreOverhang{0.5pt}
				\AxiomC{$\Gamma \provesILL \bang\varphi$}
				\AxiomC{$\Delta, \varphi \provesILL \psi$}
				\RightLabel{$\bang$-Dereliction}
				\BinaryInfC{$\Gamma, \Delta\provesILL \psi$}
		\end{prooftree}
		\begin{prooftree}
			\def\ScoreOverhang{0.5pt}
				\AxiomC{$\Gamma \provesILL \bang\varphi$}
				\AxiomC{$\Delta \provesILL \psi$}
				\RightLabel{$\bang$-Weakening}
				\BinaryInfC{$\Gamma, \Delta\provesILL \psi$}
		\end{prooftree}
		\begin{prooftree}
			\def\ScoreOverhang{0.5pt}
				\AxiomC{$\Gamma \provesILL \bang\varphi$}
				\AxiomC{$\Delta,\bang\varphi,\bang\varphi \provesILL \psi$}
				\RightLabel{$\bang$-Contraction}
				\BinaryInfC{$\Gamma, \Delta\provesILL \psi$}
		\end{prooftree}
	\end{minipage}
	\end{center}
\end{frame}
%%%%%%%%%%%%%%%%%%%%%%%%%%%%%%%%%%%%%%%%%%%%%%%%%%%%%%%%
%%%%%%%%%%%%%%%%%%%%%%%%%%%%%%%%%%%%%%%%%%%%%%%%%%%%%%%%
\begin{frame}{Alternative systems}
	\begin{itemize}
	\item One may consider other natural deduction systems.
	\pause
	\item The exponential has two rules only in the system of Negri:
	\end{itemize}
	\begin{prooftree}
	\AxiomC{$(\bang\varphi)^m,\Gamma \proves \chi$}
	\AxiomC{$\Delta_1 \proves \bang\psi_1 \dots \Delta_n \proves \bang\psi_n $}
	\AxiomC{$\bang\psi_1,\dots,\bang\psi_n \proves \varphi$}
	\RightLabel{$\bang$-I}
	\TrinaryInfC{$\Gamma,\Delta_1,\dots\Delta_n\proves\chi$}
	\end{prooftree}
	\begin{prooftree}
	\AxiomC{$\Gamma\proves\bang\varphi$}
	\AxiomC{$\Delta,\varphi \proves \psi$}
	\RightLabel{$\bang$-E}
	\BinaryInfC{$\Gamma,\Delta\proves\psi$}
	\end{prooftree}
	\end{frame}
%%%%%%%%%%%%%%%%%%%%%%%%%%%%%%%%%%%%%%%%%%%%%%%%%%%%%%%% 
%%%%%%%%%%%%%%%%%%%%%%%%%%%%%%%%%%%%%%%%%%%%%%%%%%%%%%%%
%\begin{frame}{Weakening and Contraction derivability}
%	\begin{proof}
%	It follows immediately by setting $n=1$ and $m=2$ for contraction and $m=0$ for weakening in the introduction rule, $\psi = \varphi$ and $\Delta = \bang \varphi$ that 
%	\begin{prooftree}
%	\AxiomC{$(\bang\varphi)^m, \Gamma \proves \chi$}
%	\AxiomC{$\bang\varphi \proves \bang\varphi$}
%	\AxiomC{$\bang\varphi \proves \bang\varphi$}
%	\AxiomC{$\varphi \proves \varphi$}
%	\RightLabel{$\bang$E}
%	\BinaryInfC{$\bang\varphi \proves \varphi$}
%	\RightLabel{$\bang$I}
%	\TrinaryInfC{$\Gamma,\bang \varphi\proves\chi$}
%	\end{prooftree}
%	\end{proof}
%\end{frame}
	
%%%%%%%%%%%%%%%%%%%%%%%%%%%%%%%%%%%%%%%%%%%%%%%%%%%%%%%%
%%%%%%%%%%%%%%%%%%%%%%%%%%%%%%%%%%%%%%%%%%%%%%%%%%%%%%%%
\section{Approaching substructurality in Base-extention Semantics}
\begin{frame}{Substructural atomic derivability}
\begin{definition}[Basic rules]
Basic rules take the following form: 
\[\openaddrule (\at{P}_{1_i} \seq \at{q}_{1_i})\closeaddrule^{l_1}_{i=1}, \dots, \openaddrule (\at{P}_{n_i} \seq \at{q}_{n_i})\closeaddrule^{l_n}_{i=1} \Rightarrow \at{r}\]
where 
\begin{itemize}
\item Each $P_i$ is an atomic multiset, called a premiss multiset.
\item Each $q_i$ and $r$ is an atomic proposition.
\item Each $(P_i \seq q_i)$ is a pair $(P_i, q_i)$ called an atomic sequent.
\item Each collection $\openaddrule (\at{P}_{i_1} \seq \at{q}_{i_1}), \dots, (\at{P}_{i_{l_i}} \seq \at{q}_{i_{l_i}})\closeaddrule$ is called an atomic box.
    \end{itemize}
\end{definition}
%\pause
%\emph{Given derivations of $q_{i_j}$ from $P_{i_j}$ relative to some multisets $C_i$ one infers $r$ relative to all $C_i$, discharging from the derivations in question, the premiss atomic multisets $P_{i_j}$.} 
\end{frame}
%%%%%%%%%%%%%%%%%%%%%%%%%%%%%%%%%%%%%%%%%%%%%%%%%%%%%%%%
%%%%%%%%%%%%%%%%%%%%%%%%%%%%%%%%%%%%%%%%%%%%%%%%%%%%%%%%
\begin{frame}{Substructural atomic derivability}
\begin{definition}[Basic derivability relation]
The relation of derivability in a base $\baseB$, is defined inductively as so:
\begin{itemize}
\item[Ref] $p \deriveBaseM{\baseB} p$
\item[App] Given that $(\openaddrule(\at{P}_{1_i} \seq \at{q}_{1_{i}})\Closeaddrule{i=1}{l_1},\dots,\openaddrule(\at{P}_{n_i} \seq \at{q}_{n_{i}}) \Closeaddrule{i=1}{l_n} \Rightarrow \at{r}) \in \baseB$ and atomic multisets $\at{C}_i$ such that the following hold:
        \[\at{C}_i\msetsum \at{P}_{i_j} \deriveBaseM{\baseB} \at{q}_{i_j} \text{ for all } i = 1,\dots, n \text{ and } j = 1,\dots,l_i\]
Then $\at{C}_{1}\msetsum\dots\msetsum\at{C}_{n}\deriveBaseM{\baseB} r$. 
\end{itemize}
\end{definition}
\end{frame}
%%%%%%%%%%%%%%%%%%%%%%%%%%%%%%%%%%%%%%%%%%%%%%%%%%%%%%%%
%%%%%%%%%%%%%%%%%%%%%%%%%%%%%%%%%%%%%%%%%%%%%%%%%%%%%%%%
\section{Base-extension Semantics for IMALL}
%\begin{frame}{Base-extension Semantics for IMALL}
\begin{frame}
	\noindent\begin{minipage}{0.47\textwidth}\scriptsize
	\begin{figure}[t]
		%\hrule \vspace{1mm}
				  \[
				\begin{array}{r@{\qquad}l@{\quad}c@{\quad}l}
				   \mbox{(At)} & \suppM{\baseB}{L} p  & \text{ iff } &   L\deriveBaseM{\baseB} p  \\ [1mm]
					% 
					(\mto) & \suppM{\baseB}{L} \varphi \mto \psi & \text{ iff } & \varphi \suppM{\baseB}{L} \psi \\[1mm]
					% 
					(\mand) & \suppM{\baseB}{L} \varphi \mand \psi   & \text{ iff } &  \text{for any } \baseC  \text{ such that } \baseB \baseLeq \baseC \text{, atomic multisets } K \\ 
					& & & \text{and any } p \in \At, \text{ if } \varphi\msetsum\psi\suppM{\baseB}{K}p \text{ then } \suppM{\baseC}{L\msetsum K} p  \\[1mm]
					%
					(\mtop) & \suppM{\baseB}{L} \mtop   & \text{ iff } &  \text{for any } \baseC  \text{ such that } \baseB \baseLeq \baseC \text{, atomic multisets } K \\ 
					& & & \text{and any } p \in \At, \text{ if } \suppM{\baseB}{K}p \text{ then } \suppM{\baseC}{L\msetsum K} p  \\[1mm]
					%
					(\top) & \suppM{\baseB}{L} \top & \text{ iff } & \text{always} \\[1mm] 
					%
					(\aand) & \suppM{\baseB}{L} \varphi \aand \psi   & \text{ iff } &   \suppM{\baseB}{L} \varphi \text{ and }   \suppM{\baseB}{L} \psi  \\[1mm] 
					%
					(\aor) & \suppM{\baseB}{L} \varphi \aor \psi & \text{ iff } &  \text{for any } \baseC  \text{ such that } \baseB \baseLeq \baseC \text{, atomic multisets } K \\ 
					& & & \text{and any } p \in \At, \text{ if } \varphi \suppM{\baseC}{K} p \text{ and } \psi \suppM{\baseC}{K} p, \text{ then } \suppM{\baseC}{L\msetsum K} p  \\[1mm]
					% 
					(\abot) & \suppM{\baseB}{L} \abot & \text{iff} &    \suppM{\baseB}{L} p \text{ for any } p \in \At \\[1mm]
					%
					(\bang) & \suppM{\baseB}{L} \bang \varphi & \text{iff} &  \text{for any } \baseC  \text{ such that } \baseB \baseLeq \baseC \text{, atomic multisets } K \\ 
					& & & \text{and any } p \in \At, \text{ if for any } \baseD \text{ such that } \baseC\baseLeq\baseD, \\
					& & & (\text{if } \suppM{\baseD}{\emptymultiset}\varphi \text{ then } \suppM{\baseD}{L}p) \text{ then } \suppM{\baseC}{L\msetsum K}p  \\[1mm]
					%
					(\msetsum) & \suppM{\baseB}{L}\Gamma\msetsum\Delta & \text{iff} & \text{there exists multisets } K \text{ and } M \text{ such that } L=K\msetsum M \\ 
					& & &\text{and } \suppM{\baseB}{K}\Gamma \text{ and } \suppM{\baseB}{M}\Delta \\
					%
					\mbox{(Inf)} & \hspace{-1em} \bang\Delta\msetsum\Theta \suppM{\baseB}{L} \varphi & \text{ iff } & \text{for any } \baseC \text{ such that } \baseB \baseLeq \baseC \text{, atomic multisets } K, \\
					& & & \text{if }   \suppM{\baseC}{\emptymultiset} \Delta \text{ and } \suppM{\baseC}{K} \Theta \text{ then } \suppM{\baseC}{L\msetsum K} \varphi \\[1mm]
				\end{array}
				\]
	\end{figure}
	\end{minipage}
\end{frame}
%%%%%%%%%%%%%%%%%%%%%%%%%%%%%%%%%%%%%%%%%%%%%%%%%%%%%%%%
%%%%%%%%%%%%%%%%%%%%%%%%%%%%%%%%%%%%%%%%%%%%%%%%%%%%%%%%
\begin{frame}{Notes on $\suppM{\baseB}{L}$}
	\begin{itemize}
	\item The sequent $\langle\Gamma,\varphi\rangle$ is said to be valid if and only if $\Gamma\suppM{\emptybase}{\emptymultiset}\varphi$ holds.
	\vspace{5pt}
	\item We frequently write this as $\Gamma \suppM{}{}\varphi$.
	\end{itemize}
\end{frame}
%%%%%%%%%%%%%%%%%%%%%%%%%%%%%%%%%%%%%%%%%%%%%%%%%%%%%%%%
%%%%%%%%%%%%%%%%%%%%%%%%%%%%%%%%%%%%%%%%%%%%%%%%%%%%%%%%
\begin{frame}{Notes on $\suppM{\baseB}{L}$}
	\begin{itemize}
	\item If $\suppM{\baseB}{L}\varphi$ then for all $\baseC \baseGeq \baseB$ we have $\suppM{\baseC}{L}\varphi$.
	\vspace{5pt}
	\item Given $\Gamma\suppM{\baseB}{L}\varphi$ and $\suppM{\baseB}{K}\Gamma$, then it holds that $\suppM{\baseB}{L\msetsum K}\varphi$.
	\end{itemize}
\end{frame}
%%%%%%%%%%%%%%%%%%%%%%%%%%%%%%%%%%%%%%%%%%%%%%%%%%%%%%%%
%%%%%%%%%%%%%%%%%%%%%%%%%%%%%%%%%%%%%%%%%%%%%%%%%%%%%%%%
\begin{frame}{Why the extension to Inf?}
	\begin{itemize}
		\item Up until now, all Base-extension Semantics have had a simple (Inf) clause. 
		\vspace{5pt}
		\item For example in IMALL, it suffices to take the following:\newline
		$\Gamma\suppM{\baseB}{L}\varphi$ iff for all $\baseC \baseGeq \baseB$ and $K$, $\suppM{\baseC}{K}\Gamma$ implies $\suppM{\baseC}{L\msetsum K}\varphi$
	\end{itemize}
\end{frame}
%%%%%%%%%%%%%%%%%%%%%%%%%%%%%%%%%%%%%%%%%%%%%%%%%%%%%%%%
%%%%%%%%%%%%%%%%%%%%%%%%%%%%%%%%%%%%%%%%%%%%%%%%%%%%%%%%
\begin{frame}{Why the extension to Inf?}
	\begin{center}
		\textbf{Question: Is this still an inferentialist semantics?}
	\end{center}
\end{frame}
%%%%%%%%%%%%%%%%%%%%%%%%%%%%%%%%%%%%%%%%%%%%%%%%%%%%%%%%
%%%%%%%%%%%%%%%%%%%%%%%%%%%%%%%%%%%%%%%%%%%%%%%%%%%%%%%%
\section{Including the exponential}
\begin{frame}{Can we better understand the exponential?}
\begin{itemize}	
	\item $\bang \varphi \suppM{}{} \varphi\mand\dots\mand\varphi$
	%
	\vspace{5pt}
	%
	\item $\suppM{\baseB}{L}\bang(\varphi\aand\psi)$ iff $\suppM{\baseB}{L}\bang\varphi \mand \bang\psi$
\end{itemize}
\end{frame}
%%%%%%%%%%%%%%%%%%%%%%%%%%%%%%%%%%%%%%%%%%%%%%%%%%%%%%%%
%%%%%%%%%%%%%%%%%%%%%%%%%%%%%%%%%%%%%%%%%%%%%%%%%%%%%%%%
\begin{frame}{Expanding what we know}
\begin{center}
\begin{align*}
&\bang\varphi \suppM{}{} \varphi \mand \ldots \mand \varphi \text{ iff } \\
&\text{ for all bases } \baseB \text{ and atomic multisets } L \text{, such that } \suppM{\baseB}{L} \bang\varphi \text{ then } \\ 
&\suppM{\baseB}{L} \varphi \mand \ldots \mand \varphi
\end{align*}
\end{center}
\end{frame}
%%%%%%%%%%%%%%%%%%%%%%%%%%%%%%%%%%%%%%%%%%%%%%%%%%%%%%%%
%%%%%%%%%%%%%%%%%%%%%%%%%%%%%%%%%%%%%%%%%%%%%%%%%%%%%%%%
\begin{frame}{Expanding what we know}
\begin{center}
\begin{itemize}
\item So when does $\suppM{\baseB}{L} \varphi \mand \ldots \mand \varphi$ hold?
\end{itemize}
\end{center}
\end{frame}
%%%%%%%%%%%%%%%%%%%%%%%%%%%%%%%%%%%%%%%%%%%%%%%%%%%%%%%%
\begin{frame}{Expanding what we know}
\begin{center}
Considering the case when $\varphi = \at{p}$ for simplicity we get the following:
\begin{itemize}
\pause
\item Given a base $\baseB$ such that $\deriveBaseM{\baseB}\at{p}$ and $L=\emptymultiset$.
\pause
\item Then we have that $\suppM{\baseB}{L}\at{p} \mand \ldots \mand \at{p}$.
\pause
\item Examples of such bases:
\begin{itemize}
	\item $\baseB = \{\emptymultiset \Rightarrow \at{p}\}$
	\item $\baseB = \{\emptymultiset \Rightarrow \at{q},\, \at{q}\Rightarrow \at{p} \}$
\end{itemize}
\pause
\end{itemize} 
\vspace{10pt}
Note that if $L \neq \emptymultiset$ then the only way that this relation could hold is if the atoms were derivable from within the base, i.e. if were valid $\suppM{\baseB}{\emptymultiset}L$.

So ultimately, $\suppM{\baseB}{\emptymultiset}p\mand \ldots \mand p$. 
\end{center}
\end{frame}
%%%%%%%%%%%%%%%%%%%%%%%%%%%%%%%%%%%%%%%%%%%%%%%%%%%%%%%%
\begin{frame}{Expanding what we know}
\begin{center}
\noindent
\emph{Inferring from $\bang \varphi$ should imply that $\suppM{\baseB}{\emptymultiset}\varphi$.}
\end{center}
\end{frame}
%%%%%%%%%%%%%%%%%%%%%%%%%%%%%%%%%%%%%%%%%%%%%%%%%%%%%%%%
%%%%%%%%%%%%%%%%%%%%%%%%%%%%%%%%%%%%%%%%%%%%%%%%%%%%%%%%
\begin{frame}{A proof-theoretic flavour of the $\bang$}
	We now consider the second identity: $\suppM{\baseB}{L}\bang(\varphi\aand\psi)$ iff $\suppM{\baseB}{L}\bang\varphi \mand \bang\psi$
	with $\psi = \top$. Furthermore since $\varphi\aand\top \equiv \varphi$ we go as follows:
	\begin{align*}
		\suppM{\baseB}{L}\bang\varphi &\text{ iff } \suppM{\baseB}{L}\bang\varphi \mand \bang\top \\
		&\text{ iff for all } \baseC\baseGeq\baseB, K \text{ and } p\in\At,\bang\varphi\msetsum\bang\top\suppM{\baseC}{K}p\text{ implies }\suppM{\baseC}{L\msetsum K}p\\
		&\text{ iff for all } \baseC\baseGeq\baseB, K \text{ and } p\in\At, \\
		&\hspace{0.4cm}\text{ if } (\text{ for all } \baseD\baseGeq\baseC,\text{ if }\suppM{\baseD}{\emptymultiset}\varphi\text{ and }\suppM{\baseD}{\emptymultiset}\top\text{ then }\suppM{\baseD}{K}p)\text{ then }\suppM{\baseC}{L\msetsum K}\\
		&\text{ iff for all } \baseC\baseGeq\baseB, K \text{ and } p\in\At, \\
		&\hspace{0.4cm}\text{ if } (\text{ for all }  \baseD\baseGeq\baseC,\text{ if }\suppM{\baseD}{\emptymultiset}\varphi\text{ then }\suppM{\baseD}{K}p)\text{ then }\suppM{\baseC}{L\msetsum K}
	\end{align*}
\end{frame}
%%%%%%%%%%%%%%%%%%%%%%%%%%%%%%%%%%%%%%%%%%%%%%%%%%%%%%%%
%%%%%%%%%%%%%%%%%%%%%%%%%%%%%%%%%%%%%%%%%%%%%%%%%%%%%%%%
\begin{frame}{Some failed attempts}
\begin{itemize}
\item The most convincing hypothesis was that $\suppM{\baseB}{L}\bang\varphi$ iff $L = \emptymultiset$ and for all $\baseC \baseGeq \baseB$, atomic multisets $K$ and atoms $p$, if $\varphi\suppM{\baseC}{K}p$ then $\suppM{\baseC}{K}p$.
\pause
\item This however fails to be complete as it requires us to be able to prove from $L \deriveBaseM{\baseB} \flatIMALL{(\bang \varphi)}$ that $L = \emptymultiset$, to pass Proposition 31.
\pause
\item Attempts to generalise this by replacing $L=\emptymultiset$ with $\suppM{\baseB}{\emptymultiset}L$ fail for the same reason.
\pause
\item A different attempt, based on a notion of closure was to define $\suppM{\baseB}{L}\bang \varphi$ iff for every $K, p$ and $\baseC \baseGeq \baseB$ if we have $\varphi\suppM{\baseC}{K}p$ then $\suppM{\baseC}{L,K}p$ and $\Quinequote{\forall}M(\suppM{\baseB}{M}\varphi\, \Quinequote{\Rightarrow}\suppM{\baseB}{\emptymultiset}M)$ also fails as there is no way to prove the conclusion of the second conditional.
\end{itemize}
\end{frame}
%%%%%%%%%%%%%%%%%%%%%%%%%%%%%%%%%%%%%%%%%%%%%%%%%%%%%%%%
%%%%%%%%%%%%%%%%%%%%%%%%%%%%%%%%%%%%%%%%%%%%%%%%%%%%%%%%
\begin{frame}{The idea of a zoned semantics}
\begin{center}
\begin{itemize}
\item Not absurd to distinguish basic sentences based on whether they are to be used as a hypothesis in an intuitionistic or a resourceful way. 
\pause
\item Thus we define a new type of basic rule that lookds something like this:
\begin{prooftree}
\AxiomC{$[P_1;C_1]$}
\noLine
\UnaryInfC{$\vdots$}
\noLine
\UnaryInfC{$q_1$}
\AxiomC{$\dots$}
\AxiomC{$[P_n;C_n]$}
\noLine
\UnaryInfC{$\vdots$}
\noLine
\UnaryInfC{$q_n$}
\RightLabel{$\mathcal{R}$}
\TrinaryInfC{$r$}
\end{prooftree}
\end{itemize}
\end{center}
\end{frame}
%%%%%%%%%%%%%%%%%%%%%%%%%%%%%%%%%%%%%%%%%%%%%%%%%%%%%%%%
%%%%%%%%%%%%%%%%%%%%%%%%%%%%%%%%%%%%%%%%%%%%%%%%%%%%%%%%
\begin{frame}{Exploring the zoned semantics}
	\begin{definition}[Zoned basic derivability relation]
		The relation of derivability in a base $\baseB$, is defined inductively as so:
		\begin{itemize}
		\item[Ref] $S;T \deriveBaseM{\baseB} p$ iff ($p \in S$ and $T=\emptymultiset$) or $T=[p]$.
		\item[App] Given that $(\openaddrule(P_{1_i};C_{1_i} \seq q_{1_{i}})\Closeaddrule{i=1}{l_1},\dots,\openaddrule(P_{n_i};C_{n_i} \seq q_{n_{i}}) \Closeaddrule{i=1}{l_n} \Rightarrow \at{r}) \in \baseB$ and atomic multisets $G$ and $\at{C}_i$ such that the following hold:
				\[G\msetsum P_{i_j};L_i\msetsum C_{i_j} \deriveBaseM{\baseB} \at{q}_{i_j} \text{ for all } i = 1,\dots, n \text{ and } j = 1,\dots,l_i\]
		Then $G;L_{1}\msetsum\dots\msetsum L_{n}\deriveBaseM{\baseB} r$. 
		\end{itemize}
	\end{definition}
\end{frame}
%%%%%%%%%%%%%%%%%%%%%%%%%%%%%%%%%%%%%%%%%%%%%%%%%%%%%%%%
%%%%%%%%%%%%%%%%%%%%%%%%%%%%%%%%%%%%%%%%%%%%%%%%%%%%%%%%
\begin{frame}{Exploring the zoned semantics}
\begin{itemize}
\item $\suppM{\baseB}{G;L}p$ iff $G;L\deriveBaseM{\baseB}p$
\vspace{0.3cm}
\item $\suppM{\baseB}{G;L}\Gamma\msetsum\Delta$ iff there exists $L=U\msetsum V$ such that $\suppM{\baseB}{G;U}\Gamma$ and $\suppM{\baseB}{G;V}\Delta$
\vspace{0.3cm}
\item $\suppM{\baseB}{G;L} \bang\varphi$ iff for every base $\baseC \baseGeq \baseB$, atomic multisets $h,K$ and atom $p,$ such that $\varphi;\cdot \suppM{\baseC}{H; K} p$ hold, then $\suppM{\baseC}{G\msetsum H; L\msetsum K}p$.
\vspace{0.3cm}
\item $\Gamma;\Delta\suppM{\baseB}{G;L}\varphi$ iff for every $\baseC \baseGeq \baseB$ and atomic multisets $H,K$ such that $\suppM{\baseC}{H;\emptymultiset}\Gamma$ and $\suppM{\baseC}{G\msetsum H;L\msetsum K}\Delta$ then $\suppM{\baseC}{G\msetsum H;L\msetsum K}\varphi$
\end{itemize}
\end{frame}
%%%%%%%%%%%%%%%%%%%%%%%%%%%%%%%%%%%%%%%%%%%%%%%%%%%%%%%%
%%%%%%%%%%%%%%%%%%%%%%%%%%%%%%%%%%%%%%%%%%%%%%%%%%%%%%%%
\begin{frame}{Some other work undertaken}
	\begin{itemize}
		\item Investigated alternative IPL support relations that are more ``resource''-aware.
		\item Started investigating translations between Sandqvist's semantics for IPL and the semantics presented here for ILL.
		\item Very recently, I have been working on a Base-extension semantics for Lambek Calculus.
	\end{itemize}
\end{frame}
%%%%%%%%%%%%%%%%%%%%%%%%%%%%%%%%%%%%%%%%%%%%%%%%%%%%%%%%
\begin{frame}{Future directions of work}
	\begin{itemize}
		\item Study how rules are generally translated from Sandqvists bases for IPL to the bases I presented here.
		\item Give a Proof-theoretic Semantics for Classical Linear Logic.
		\item Similarly, study other similar logics such as DILL, FILL and what happens when we have subexponentials.
		\item Give a categorical interpretation of the semantics presented herein.
		\item Investigate the connection to Linear Logic programming.
		\item Develop a connection to the Kripke semantics of Miller for his fragment of ILL.
	\end{itemize}
\end{frame}
%%%%%%%%%%%%%%%%%%%%%%%%%%%%%%%%%%%%%%%%%%%%%%%%%%%%%%%%
\begin{frame}{Thank you!}
\begin{center}
Thank you for listening!
\end{center}
\end{frame}
\end{document}
