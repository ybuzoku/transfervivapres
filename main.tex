\documentclass{beamer}
\usetheme{ucl}

\newcommand\hmmax{0}
\newcommand\bmmax{0}

\usepackage{graphicx}
\usepackage[all]{xy}

% \usepackage[utf8]{inputenc}
\usepackage{amsmath,amsthm,stmaryrd,textcomp,amssymb,enumerate}
\usepackage{mathbbol}
\usepackage{bbold}
\usepackage{dsfont}
\usepackage{algorithm} % Boxes/formatting around algorithms
\usepackage{mathtools}
%\usepackage[noend]{algpseudocode} %
\usepackage{hyperref}
\usepackage{xcolor}
\hypersetup{
    colorlinks=true,
    urlcolor     = blue, %Colour for external hyperlinks
    linkcolor    = blue, %Colour of internal links
    citecolor    = blue %Colour of citations
}
% \usepackage{amsthm}
\usepackage{comment} % To comment paragraphs.
\usepackage{color}
\usepackage{relsize}
\usepackage{bm}
\usepackage{bbm} % For mathbb numbers.
\usepackage{subcaption} % Subfigure.
\usepackage{cleveref}
\usepackage{bussproofs}
%\usepackage{ebproof}
\usepackage{pifont}
\usepackage{wrapfig}
\usepackage{cutwin}

\usepackage{tikz-cd}
\usepackage{tikzit}
\input{local.tikzstyles}
\usepackage{mathrsfs}
%\usepackage{proof}
% NEW IMPORTS
\usepackage{cmll}       % For \with
\newenvironment{bprooftree}
  {\leavevmode\hbox\bgroup}
  {\DisplayProof\egroup}
  
%%%%%%%%%%%%%% Define the new environments here;
%\newenvironment{qparts}{\begin{enumerate}[{(}a{)}]}{\end{enumerate}}
% \def\endproofmark{$\Box$}
%\renewenvironment{proof}{\par{\bf Proof}:}{\hfill$\blacksquare$}

\newtheorem{defn}[theorem]{Definition}

\newtheorem{question}[theorem]{Question}
\newtheorem{remark}[theorem]{Remark}
\newtheorem{construction}[theorem]{Construction}
\newtheorem{observation}[theorem]{Observation}

%Make font sans-serif
\renewcommand{\familydefault}{\sfdefault}

%%%%%% Define commands here;

% Basic logic notation.
%\newcommand{\seq}{\triangleright}
\newcommand{\seq}{\rightsquigarrow }
\newcommand{\sequent}[2]{{#1} \seq {#2}}
\newcommand{\thmSequent}[2]{{#1} \vdash {#2}}

%%%%%%%%%%%%%%%%%%%%%%%%%%%%%%%%%%%%%%%%%
%       BeS
%%%%%%%%%%%%%%%%%%%%%%%%%%%%%%%%%%%%%%%%%
\newcommand{\base}[1]{\mathscr{#1}}
\newcommand{\baseB}{\base{B}}
\newcommand{\baseC}{\base{C}}
\newcommand{\baseD}{\base{D}}
\newcommand{\baseE}{\base{E}}
\newcommand{\baseF}{\base{F}}
\newcommand{\baseG}{\base{G}}
\newcommand{\baseH}{\base{H}}
\newcommand{\baseX}{\base{X}}
\newcommand{\baseY}{\base{Y}}
\newcommand{\baseZ}{\base{Z}}
\newcommand{\baseIMALL}{\base{N}}
\newcommand{\emptybase}{\varnothing}
\newcommand{\basis}[1]{\mathfrak{#1}}
% \newcommand{\at}[1]{\mathrm{#1}}
\newcommand{\at}[1]{{#1}}
\newcommand{\At}{\mathrm{At}}

%\newcommand{\baseGeq}{\succeq}
%\newcommand{\baseLeq}{\preceq}
\newcommand{\baseGeq}{\supseteq}
\newcommand{\baseLeq}{\subseteq}

\newcommand{\suppTor}[1]{\Vdash_{\!\!#1}}
% \newcommand{\supp}[1]{\Vdash_{\!\!#1}}
% \newcommand{\suppNew}[1]{\vDash_{\!#1}}
\newcommand{\suppNew}[1]{\Vdash^{*}_{\!#1}}
\newcommand{\entails}{\Vdash}
%\newcommand{\proves}[1]{\vdash_{\!\!#1}}
\newcommand{\proves}{\vdash}
\newcommand{\plus}{+}
\newcommand{\system}[1]{\mathsf{#1}}
\newcommand{\Atoms}{\set{A}}
\newcommand{\Formulas}{\set{F}}
\newcommand{\Bunches}{\set{B}}
\newcommand{\set}[1]{\mathbb{#1}}


% ILL
\newcommand{\IPL}{IPL}
\newcommand{\IMLL}{IMLL}
\newcommand{\IMALL}{IMALL}
\newcommand{\ILL}{ILL}
\newcommand{\provesIMALL}{\vdash_{\mathrm{IMALL}}}
\newcommand{\provesILL}{\vdash_{\mathrm{ILL}}}
\newcommand{\multiset}{\mathcal{M}}
\newcommand{\fmultiset}{\mathcal{M}_{\!{f}}}
\newcommand{\emptymultiset}{\varnothing}
\newcommand{\deriveBaseM}[1]{\vdash_{\!\!#1}}
\newcommand{\suppM}[2]{\Vdash_{ \!\!#1 }^{ \!\!#2 }}
\newcommand{\suppMAlt}[2]{\Vvdash_{ \!\!#1 }^{ \!\!#2 }}
\newcommand{\IMALLformula}{\mathsf{Form}}
\newcommand{\mand}{\otimes}
\newcommand{\mtop}{1}
\newcommand{\mto}{\multimap}
\newcommand{\aand}{\with}
\newcommand{\aor}{\oplus}
\newcommand{\abot}{0}
\newcommand{\msetunion}{,}
\newcommand{\bang}{\mathop{!}}
\newcommand{\makeMultiset}[1]{[#1]}
\newcommand{\flatIMALL}[1]{{#1}^{\flat}}
\newcommand{\deflatIMALL}[1]{{#1}^{\natural}}
\newcommand{\openaddrule}{\{}
\newcommand{\closeaddrule}{\}}
\newcommand{\Closeaddrule}[2]{\}^{#2}_{#1}}
\newcommand{\extAt}{\widetilde{\At}}
\DeclareMathSymbol{\fatcomma}{\mathrel}{bbold}{\lq\,}

%Misc
\newbox\qqBoxA
\newdimen\qqCornerHgt
\setbox\qqBoxA=\hbox{$\ulcorner$}
\global\qqCornerHgt=\ht\qqBoxA
\newdimen\qqArgHgt
\def\Quinequote #1{%
    \setbox\qqBoxA=\hbox{$#1$}%
    \qqArgHgt=\ht\qqBoxA%
    \ifnum     \qqArgHgt<\qqCornerHgt \qqArgHgt=0pt%
    \else \advance \qqArgHgt by -\qqCornerHgt%
    \fi \raise\qqArgHgt\hbox{$\ulcorner$} \box\qqBoxA %
    \raise\qqArgHgt\hbox{$\urcorner$}}



\setbeamersize{description width=2em}
%%% Remove nav symbols (and shift any logo down to corner)
\setbeamertemplate{navigation symbols}{\vspace{-2ex}}
\setbeamertemplate{footline}[author title date]
\setbeamercolor{banner}{bg=darkpurple}
\useinnertheme{blockborder}

\graphicspath{{./images/} }
\title[P-tS for ILL]{Transfer viva: \newline Proof-theoretic Semantics for ILL and beyond}
%\author{\texorpdfstring{Yll Buzoku\newline\url{y.buzoku@ucl.ac.uk}}{https://www.homepages.ucl.ac.uk/~zcapybu/}}
\author{Yll Buzoku}
\institute[UCL]{%
  Department of Computer Science \\ %
  University College London
}
\date{May 30, 2024}

%%%%% Uncomment the bottom 7 lines to introduce between ToC between sections
%\AtBeginSection[]
%{
%  \begin{frame}
%    \frametitle{Presentation root directory}
%    \tableofcontents[currentsection]
%  \end{frame}
%}

\begin{document}
%%%%%%%%%%%%%%%%%%%%%%%%%%%%%%%%%%%%%%%%%%%%%%%%%%%%%%%%
%%%%%%%%%%%%%%%%%%%%%%%%%%%%%%%%%%%%%%%%%%%%%%%%%%%%%%%%
\begin{frame}
\titlepage
\end{frame}
%%%%%%%%%%%%%%%%%%%%%%%%%%%%%%%%%%%%%%%%%%%%%%%%%%%%%%%%
%%%%%%%%%%%%%%%%%%%%%%%%%%%%%%%%%%%%%%%%%%%%%%%%%%%%%%%%
\section*{Goals for this presentation}
\begin{frame}{Goals for this talk}
\begin{itemize}
\item To cover what work I have already done in the direction obtaining a BeS for IMALL.
\item To cover some issues with the current approach to dealing with the modality of ILL.
\item To discuss a (currently promising) approach to dealing with the modality of ILL.
\end{itemize}
\end{frame}
%%%%%%%%%%%%%%%%%%%%%%%%%%%%%%%%%%%%%%%%%%%%%%%%%%%%%%%%
%%%%%%%%%%%%%%%%%%%%%%%%%%%%%%%%%%%%%%%%%%%%%%%%%%%%%%%%
\begin{frame}{A natural deduction system for IMALL}
\begin{center}
\noindent\begin{minipage}{0.47\textwidth}\scriptsize
\begin{prooftree}
  \def\ScoreOverhang{0.5pt}
  	\AxiomC{}
  	\RightLabel{Ax}
  	\UnaryInfC{$\varphi\proves\varphi$}
\end{prooftree}
\end{minipage}
\vspace{2pt}
\noindent\begin{minipage}{0.47\textwidth}\scriptsize
	\begin{prooftree}
  	\def\ScoreOverhang{0.5pt}
  		\AxiomC{$\Gamma,\varphi \proves \psi$}
  		\RightLabel{$\mto$-I}
  		\UnaryInfC{$\Gamma \proves \varphi \mto \psi$}
	\end{prooftree}
	\begin{prooftree}
	\def\ScoreOverhang{0.5pt}
		\AxiomC{$\Gamma \proves \varphi$}
		\AxiomC{$\Delta \proves \psi$}
		\RightLabel{$\mand$-I}
		\BinaryInfC{$\Gamma,\Delta \proves \varphi\mand\psi$}
	\end{prooftree}
	\begin{prooftree}
	\def\ScoreOverhang{0.5pt}
		\AxiomC{}
		\RightLabel{$\mtop$-I}
		\UnaryInfC{$\proves \mtop$}
	\end{prooftree}
	\begin{prooftree}
	\def\ScoreOverhang{0.5pt}
		\AxiomC{$\Gamma^{\gamma} \proves \varphi$}
		\AxiomC{$\Gamma^{\gamma} \proves \psi$}
		\RightLabel{$\aand$-I}
		\BinaryInfC{$\Gamma \proves \varphi \aand \psi$}
	\end{prooftree}
	\begin{prooftree}
	\def\ScoreOverhang{0.5pt}
		\AxiomC{$\Gamma \proves \varphi_i$}
		\RightLabel{$\aor$-$\text{I}_i$}
		\UnaryInfC{$\Gamma \proves \varphi_0 \aor \varphi_1$}
	\end{prooftree}
	\begin{prooftree}
	\def\ScoreOverhang{0.5pt}
		\AxiomC{No $\abot$ intro rule}
	\end{prooftree}
\end{minipage}%%%%%%%%%%%%%%%%%%%%%%%%%%%%%%%%
\begin{minipage}{0.47\textwidth}\scriptsize
	\begin{prooftree}
  	\def\ScoreOverhang{0.5pt}
  		\AxiomC{$\Gamma \proves \varphi \mto \psi$}
  		\AxiomC{$\Delta \proves \varphi$}
  		\RightLabel{$\mto$-E}
  		\BinaryInfC{$\Gamma,\Delta \proves \psi$}
	\end{prooftree}
	\begin{prooftree}
	\def\ScoreOverhang{0.5pt}
		\AxiomC{$\Gamma \proves \varphi\mand\psi$}
		\AxiomC{$\Delta, \varphi, \psi \proves \chi$}
		\RightLabel{$\mand$-E}
		\BinaryInfC{$\Gamma, \Delta \proves \chi$}
	\end{prooftree}
	\begin{prooftree}
	\def\ScoreOverhang{0.5pt}
		\AxiomC{$\Gamma \proves \varphi$}
		\AxiomC{$\Delta \proves \mtop$}
		\RightLabel{$\mtop$-E}
		\BinaryInfC{$\Gamma,\Delta \proves \varphi$}
	\end{prooftree}
	\begin{prooftree}
	\def\ScoreOverhang{0.5pt}
		\AxiomC{$\Gamma \proves \varphi_0 \aand \varphi_1$}
		\RightLabel{$\aand$-$\text{E}_i$}
		\UnaryInfC{$\Gamma \proves \varphi_i$}
	\end{prooftree}
	\begin{prooftree}
	\def\ScoreOverhang{0.5pt}
		\AxiomC{$\Gamma \proves \varphi \aor \psi$}
		\AxiomC{$\Delta^{\gamma},\varphi \proves \chi$}
		\AxiomC{$\Delta^{\gamma},\psi \proves \chi$}
		\RightLabel{$\aor$-E}
		\TrinaryInfC{$\Gamma, \Delta \proves \chi$}
	\end{prooftree}
	\begin{prooftree}
	\def\ScoreOverhang{0.5pt}
		\AxiomC{$\Gamma \proves \abot$}
		\RightLabel{$\abot$-E}
		\UnaryInfC{$\Gamma \proves \varphi$}
	\end{prooftree}
\end{minipage}
\end{center}
\end{frame}
%%%%%%%%%%%%%%%%%%%%%%%%%%%%%%%%%%%%%%%%%%%%%%%%%%%%%%%%
%%%%%%%%%%%%%%%%%%%%%%%%%%%%%%%%%%%%%%%%%%%%%%%%%%%%%%%%
\section{Approaching substructurality in Base-extention Semantics}
\begin{frame}
\begin{definition}{Atomic rules}
Basic rules take the following form: 
\[\openaddrule (\at{P}_{1_i} \seq \at{q}_{1_i})\closeaddrule^{l_1}_{i=1}, \dots, \openaddrule (\at{P}_{n_i} \seq \at{q}_{n_i})\closeaddrule^{l_n}_{i=1} \Rightarrow \at{r}\]
where 
\begin{itemize}
\item Each $P_i$ is an atomic multiset, called a premiss multiset.
\item Each $q_i$ and $r$ is an atomic proposition.
\item Each $(P_i \seq q_i)$ is a pair $(P_i, q_i)$ called an atomic sequent.
\item Each collection $\openaddrule (\at{P}_{i_1} \seq \at{q}_{i_1}), \dots, (\at{P}_{i_{l_i}} \seq \at{q}_{i_{l_i}})\closeaddrule$ is called an atomic box.
    \end{itemize}
\end{definition}
Basic rules are defined as before except now they are not allowed to take take the values $\mtop$ or $\abot$.
%\pause
%\emph{Given derivations of $q_{i_j}$ from $P_{i_j}$ relative to some multisets $C_i$ one infers $r$ relative to all $C_i$, discharging from the derivations in question, the premiss atomic multisets $P_{i_j}$.} 
\end{frame}
%%%%%%%%%%%%%%%%%%%%%%%%%%%%%%%%%%%%%%%%%%%%%%%%%%%%%%%%
%%%%%%%%%%%%%%%%%%%%%%%%%%%%%%%%%%%%%%%%%%%%%%%%%%%%%%%%
\begin{frame}
\begin{definition}[Basic derivability relation]
The relation of derivability in a base $\baseB$, is defined inductively as so:
\begin{itemize}
\item[Ref] $p \deriveBaseM{\baseB} p$
\item[App] Given that $(\openaddrule(\at{P}_{1_i} \seq \at{q}_{1_{i}})\Closeaddrule{i=1}{l_1},\dots,\openaddrule(\at{P}_{n_i} \seq \at{q}_{n_{i}}) \Closeaddrule{i=1}{l_n} \Rightarrow \at{r}) \in \baseB$ and atomic multisets $\at{C}_i$ such that the following hold:
        \[\at{C}_i, \at{P}_{i_j} \deriveBaseM{\baseB} \at{q}_{i_j} \text{ for all } i = 1,\dots, n \text{ and } j = 1,\dots,l_i\]
Then $\at{C}_{1},\dots,\at{C}_{n}\deriveBaseM{\baseB} r$. 
\end{itemize}
\end{definition}
\end{frame}
%%%%%%%%%%%%%%%%%%%%%%%%%%%%%%%%%%%%%%%%%%%%%%%%%%%%%%%%
%%%%%%%%%%%%%%%%%%%%%%%%%%%%%%%%%%%%%%%%%%%%%%%%%%%%%%%%
\section{Base-extension Semantics for IMALL}
\begin{frame}{Base-extension Semantics for IMALL}
\begin{center}
\begin{itemize}
\item[(At)] $\suppM{\baseB}{\at{L}} \at{p}$ iff $\at{L} \deriveBaseM{\baseB} \at{p}$. 
\item[($\mand$)] 
        $\suppM{\baseB}{\at{L}} \varphi \mand \psi$ iff for every $\baseC \baseGeq \baseB$, atomic multiset $\at{K}$, and atom $\at{p}$, if $\varphi\fatcomma\psi\suppM{\baseC}{\at{K}} \at{p}$ then $\suppM{\baseC}{\at{L},\at{K}} \at{p}$. 
        % 
\item[($\mtop$)] $\suppM{\baseB}{\at{L}} \mtop$ iff for every $\baseC \baseGeq \baseB$, atomic multiset $\at{K}$, and atom $\at{p}$, if $\suppM{\baseC}{\at{K}} \at{p}$, then $\suppM{\baseC}{\at{L},\at{K}} \at{p}$. 
        % 
\item[($\mto$)] $\suppM{\baseB}{\at{L}} \varphi \mto \psi$ iff $\varphi \suppM{\baseB}{\at{L}} \psi$. 
        % 
\item[($\aand$)] $\suppM{\baseB}{\at{L}} \varphi \aand \psi$ iff $\suppM{\baseB}{\at{L}} \varphi$ and $\suppM{\baseB}{\at{L}} \psi$. 
        %
\item[($\aor$)] 
        $\suppM{\baseB}{\at{L}} \varphi \aor \psi$ iff for every $\baseC \baseGeq \baseB$, atom $\at{p}$ and atomic multiset $\at{K}$ such that $\varphi \suppM{\baseC}{\at{K}} \at{p}$ and $\psi \suppM{\baseC}{\at{K}} \at{p}$ hold, then $\suppM{\baseC}{\at{L},\at{K}} \at{p}$. 
        %
\item[($\abot$)] $\suppM{\baseB}{\at{L}} \abot$ iff $\suppM{\baseB}{\at{L}} \at{p}$, for all atomic $\at{p}$.
        %
\item[($\fatcomma$)] $\suppM{\baseB}{\at{L}}\Gamma\fatcomma\Delta$ iff there are atomic multisets $\at{U}$ and $\at{V}$ such that $\at{L}=\at{K}\fatcomma\at{M}$ and that $\suppM{\baseB}{\at{K}}\Gamma$ and $\suppM{\baseB}{\at{M}}\Delta$.
        %
\item[(Inf)] For $\Theta$ a nonempty multiset: $\Theta \suppM{\baseB}{\at{L}}\varphi$ iff for all $\baseC \baseGeq \baseB$, atomic multisets $\at{K}$ and atoms $\at{p}$ if $\suppM{\baseC}{\at{K}}\Theta$ then $\suppM{\baseC}{\at{L},\at{K}}\varphi$.
\end{itemize}
\end{center}
\end{frame}
%%%%%%%%%%%%%%%%%%%%%%%%%%%%%%%%%%%%%%%%%%%%%%%%%%%%%%%%
%%%%%%%%%%%%%%%%%%%%%%%%%%%%%%%%%%%%%%%%%%%%%%%%%%%%%%%%
\section{Including the exponential}
\begin{frame}{What of the exponential?}
The exponential behaves very weirdly! Consider the rules in the sequent calculus:

\noindent\begin{minipage}{0.20\textwidth}\scriptsize
  \begin{prooftree}
  \def\ScoreOverhang{1pt}
  	\AxiomC{$\Gamma, \varphi \supset \psi$}
  	\RightLabel{$\bang$-d}
  	\UnaryInfC{$\Gamma,\bang\varphi \supset \psi$}
  \end{prooftree}
\end{minipage}
\begin{minipage}{0.20\textwidth}\scriptsize
  \begin{prooftree}
  \def\ScoreOverhang{1pt}
  	\AxiomC{$\Gamma,\bang\varphi,\bang\varphi \supset \psi$}
  	\RightLabel{$\bang$-c}
  	\UnaryInfC{$\Gamma,\bang\varphi \supset \psi$}
  \end{prooftree}
\end{minipage}
\begin{minipage}{0.20\textwidth}\scriptsize
   \begin{prooftree}
  \def\ScoreOverhang{1pt}
  	\AxiomC{$\bang\Gamma \supset\varphi$}
  	\RightLabel{$\bang$-r}
  	\UnaryInfC{$\bang\Gamma \supset \bang \varphi$}
  \end{prooftree}
\end{minipage}
\begin{minipage}{0.20\textwidth}\scriptsize
  \begin{prooftree}
  \def\ScoreOverhang{1pt}
  	\AxiomC{$\Gamma\supset\psi$}
  	\RightLabel{$\bang$-w}
  	\UnaryInfC{$\Gamma,\bang\varphi \supset \psi$}
  \end{prooftree}
\end{minipage}
\newline
\begin{itemize}
\item All "structural" aspects of the exponential are on the \emph{left} hand side.
\item The exponential right rule, still depends fully on the structure of the left hand side.
\item None of these rules are invertible, so none are particularly nice candidates for giving a definition. 
\end{itemize}
However, this means that to pinpoint the behaviour of the exponential, we need to pinpoint something about the resources which can be used to support it! 
\end{frame}
%%%%%%%%%%%%%%%%%%%%%%%%%%%%%%%%%%%%%%%%%%%%%%%%%%%%%%%%
%%%%%%%%%%%%%%%%%%%%%%%%%%%%%%%%%%%%%%%%%%%%%%%%%%%%%%%%
\begin{frame}{What of the exponential?}
\begin{itemize}
\item The only known good natural deduction system for ILL has an introduction rule for the exponential which is closer to being a necessitation rule.
\item In this system, \emph{all} connectives have generalised elimination rules.
\end{itemize}
\begin{prooftree}
\AxiomC{$(\bang\varphi)^m,\Gamma \proves \chi$}
\AxiomC{$\Delta_1 \proves \bang\psi_1 \dots \Delta_n \proves \bang\psi_n $}
\AxiomC{$\bang\psi_1,\dots,\bang\psi_n \proves \varphi$}
\RightLabel{$\bang$-I}
\TrinaryInfC{$\Gamma,\Delta_1,\dots\Delta_n\proves\chi$}
\end{prooftree}
\begin{prooftree}
\AxiomC{$\Gamma\proves\bang\varphi$}
\AxiomC{$\Delta,\varphi \proves \psi$}
\RightLabel{$\bang$-E}
\BinaryInfC{$\Gamma,\Delta\proves\psi$}
\end{prooftree}
\end{frame}
%%%%%%%%%%%%%%%%%%%%%%%%%%%%%%%%%%%%%%%%%%%%%%%%%%%%%%%%
%%%%%%%%%%%%%%%%%%%%%%%%%%%%%%%%%%%%%%%%%%%%%%%%%%%%%%%%
\begin{frame}{Weakening and Contraction derivability}
\begin{proof}
It follows immediately by setting $n=1$ and $m=2$ for contraction and $m=0$ for weakening in the introduction rule, $\psi = \varphi$ and $\Delta = \bang \varphi$ that 
\begin{prooftree}
\AxiomC{$(\bang\varphi)^m, \Gamma \proves \chi$}
\AxiomC{$\bang\varphi \proves \bang\varphi$}
\AxiomC{$\bang\varphi \proves \bang\varphi$}
\AxiomC{$\varphi \proves \varphi$}
\RightLabel{$\bang$E}
\BinaryInfC{$\bang\varphi \proves \varphi$}
\RightLabel{$\bang$I}
\TrinaryInfC{$\Gamma,\bang \varphi\proves\chi$}
\end{prooftree}
\end{proof}

\end{frame}
%%%%%%%%%%%%%%%%%%%%%%%%%%%%%%%%%%%%%%%%%%%%%%%%%%%%%%%%
%%%%%%%%%%%%%%%%%%%%%%%%%%%%%%%%%%%%%%%%%%%%%%%%%%%%%%%%
\begin{frame}{Expanding what we know}
\begin{center}
\begin{align*}
&\bang\varphi \suppM{}{} \varphi \mand \ldots \mand \varphi \text{ iff } \\
&\text{ for all bases } \baseB \text{ and atomic multisets } \at{L} \text{, such that } \suppM{\baseB}{\at{L}} \bang\varphi \text{ then } \\ 
&\suppM{\baseB}{\at{L}} \varphi \mand \ldots \mand \varphi
\end{align*}
\end{center}
\end{frame}
%%%%%%%%%%%%%%%%%%%%%%%%%%%%%%%%%%%%%%%%%%%%%%%%%%%%%%%%
%%%%%%%%%%%%%%%%%%%%%%%%%%%%%%%%%%%%%%%%%%%%%%%%%%%%%%%%
\begin{frame}{Expanding what we know}
\begin{center}
\begin{itemize}
\item So when does $\suppM{\baseB}{\at{L}} \varphi \mand \ldots \mand \varphi$ hold?
\end{itemize}
\end{center}
\end{frame}
%%%%%%%%%%%%%%%%%%%%%%%%%%%%%%%%%%%%%%%%%%%%%%%%%%%%%%%%
\begin{frame}{Expanding what we know}
\begin{center}
Considering the case when $\varphi = \at{p}$ for simplicity we get the following:
\begin{itemize}
\pause
\item Given a base $\baseB$ such that $\deriveBaseM{\baseB}\at{p}$ and $\at{L}=\emptymultiset$.
\pause
\item Then we have that $\suppM{\baseB}{\at{L}}\at{p} \mand \ldots \mand \at{p}$.
\pause
\item Examples of such bases:
\begin{itemize}
	\item $\baseB = \{\emptymultiset \Rightarrow \at{p}\}$
	\item $\baseB = \{\emptymultiset \Rightarrow \at{q},\, \at{q}\Rightarrow \at{p} \}$
\end{itemize}
\pause
\end{itemize} 
\vspace{10pt}
Note that if $L \neq \emptymultiset$ then the only way that this relation could hold is if the atoms were derivable from within the base, i.e. if were valid $\suppM{\baseB}{}L$.

So by cut, $\suppM{\baseB}{}p\mand \ldots \mand p$. 
\end{center}
\end{frame}
%%%%%%%%%%%%%%%%%%%%%%%%%%%%%%%%%%%%%%%%%%%%%%%%%%%%%%%%
\begin{frame}{Expanding what we know}
\begin{center}
\noindent
\emph{The meaning of $\suppM{\baseB}{\at{L}}\bang \varphi$ should imply that $\suppM{\baseB}{\emptymultiset}\varphi$.}
\end{center}
\end{frame}
%%%%%%%%%%%%%%%%%%%%%%%%%%%%%%%%%%%%%%%%%%%%%%%%%%%%%%%%
\begin{frame}{Exploring the suggestion}
\begin{itemize}
\item Cannot demand that $\deriveBaseM{\baseB}\varphi$. 
\pause
\item It also says that $\bang \varphi$ is derivable from axioms in the base, which corresponds to the view of $\bang\varphi$ given by Girard. %\cite{girard_1995}.
\pause
\item The most convincing hypothesis was that $\suppM{\baseB}{\at{L}}\bang\varphi$ iff $\at{L} = \emptymultiset$ and for all $\baseC \baseGeq \baseB$, atomic multisets $K$ and atoms $p$, if $\varphi\suppM{\baseC}{K}p$ then $\suppM{\baseC}{K}p$.
\pause
\item This however fails to be complete as it requires us to be able to prove from $L \deriveBaseM{\baseB} \flatIMALL{(\bang \varphi)}$ that $L = \emptymultiset$, to pass the $\dagger$ lemma.
\pause
\item Add a side condition? 
\end{itemize}
\end{frame}
%%%%%%%%%%%%%%%%%%%%%%%%%%%%%%%%%%%%%%%%%%%%%%%%%%%%%%%%
\begin{frame}{Generalisations of the suggestion}
\begin{itemize}
\item Attempts to generalise this by replacing $L=\emptymultiset$ with $\suppM{\baseB}{\emptymultiset}L$ fail for the same reason.
\vspace{0.15cm}
\item Attempts at sidestepping this by adding $\suppM{\baseB}{\emptymultiset}L$ as the hypothesis of a conditional similarly fail.
\vspace{0.15cm}
\item A different attempt, based on a notion of closure was to define $\suppM{\baseB}{L}\bang \varphi$ iff for every $K, p$ and $\baseC \baseGeq \baseB$ if we have $\varphi\suppM{\baseC}{K}p$ then $\suppM{\baseC}{L,K}p$ and $\Quinequote{\forall}M(\suppM{\baseB}{M}\varphi\, \Quinequote{\Rightarrow}\suppM{\baseB}{\emptymultiset}M)$ also fails as there is no way to prove the conclusion of the second conditional.
\end{itemize}
\end{frame}
%%%%%%%%%%%%%%%%%%%%%%%%%%%%%%%%%%%%%%%%%%%%%%%%%%%%%%%%
\begin{frame}{Possible alternative approach}
\begin{center}
\begin{itemize}
\item We want to re-introduce, in a controlled manner, the ability for normal intuitionistic inferences to be held valid. 
\pause
\item Not absurd to distinguish basic sentences based on whether they are to be used as a hypothesis in an intuitionistic or a resourceful way. 
\pause
\item The distinction \emph{must} however be made by the formula consequence relation.
\pause
\item Thus a basic rule might look something like the following:
\begin{prooftree}
\AxiomC{$[P_1;C_1]$}
\noLine
\UnaryInfC{$\vdots$}
\noLine
\UnaryInfC{$q_1$}
\AxiomC{$\dots$}
\AxiomC{$[P_n;C_n]$}
\noLine
\UnaryInfC{$\vdots$}
\noLine
\UnaryInfC{$q_n$}
\RightLabel{$\mathcal{R}$}
\TrinaryInfC{$r$}
\end{prooftree}
\end{itemize}
\end{center}
\end{frame}
%%%%%%%%%%%%%%%%%%%%%%%%%%%%%%%%%%%%%%%%%%%%%%%%%%%%%%%%
\begin{frame}{Exploring the alternative approach}
	\begin{definition}[Extended basic derivability relation]
		The relation of derivability in a base $\baseB$, is defined inductively as so:
		\begin{itemize}
		\item[Ref] $S;T \deriveBaseM{\baseB} p$ iff ($p \in S$ and $T=\emptymultiset$) or $T=[p]$.
		\item[App] Given that $(\openaddrule(P_{1_i};C_{1_i} \seq q_{1_{i}})\Closeaddrule{i=1}{l_1},\dots,\openaddrule(P_{n_i};C_{n_i} \seq q_{n_{i}}) \Closeaddrule{i=1}{l_n} \Rightarrow \at{r}) \in \baseB$ and atomic multisets $G$ and $\at{C}_i$ such that the following hold:
				\[F,P_{i_j};L_i, C_{i_j} \deriveBaseM{\baseB} \at{q}_{i_j} \text{ for all } i = 1,\dots, n \text{ and } j = 1,\dots,l_i\]
		Then $F;L_{1},\dots,L_{n}\deriveBaseM{\baseB} r$. 
		\end{itemize}
	\end{definition}
\end{frame}
%%%%%%%%%%%%%%%%%%%%%%%%%%%%%%%%%%%%%%%%%%%%%%%%%%%%%%%%
\begin{frame}{Exploring the alternative approach}
\begin{itemize}
\item $\suppM{\baseB}{F;L}p$ iff $F;L\deriveBaseM{\baseB}p$
\vspace{0.3cm}
\item $\suppM{\baseB}{F;L}\Gamma\fatcomma\Delta$ iff there exists $L=U\fatcomma V$ such that $\suppM{\baseB}{F;U}\Gamma$ and $\suppM{\baseB}{F;V}\Delta$
\pause
\vspace{0.3cm}
\item Perhaps something like this: \newline
$\suppM{\baseB}{F;\cdot}\bang \varphi$ iff for all $\baseC \baseGeq \baseB$, atomic multisets $L,K$ and atoms $p$, if $\varphi\suppM{\baseC}{L;K}p$ then $\suppM{\baseC}{F,L;K}p$ 
\end{itemize}
\end{frame}
%%%%%%%%%%%%%%%%%%%%%%%%%%%%%%%%%%%%%%%%%%%%%%%%%%%%%%%%
\begin{frame}{Exploring the alternative approach}
\begin{itemize}
\item This approach seems to provide a nice generalisation.
\vspace{0.3cm}
\item Unknown if the atomic system is sufficiently well behaved yet.
\vspace{0.3cm}
\item Seems to suggest a want for a Lolli-like encoding of sequents i.e. $(\bang \Gamma, \Delta: \varphi) \mapsto \Gamma;\Delta \Vdash \varphi$, as it is still difficult to encode the $\bang$I rule.
\vspace{0.3cm}
\item Change to a ND system with explicit structurality or a ND-like system with no explicit structural rules?
\end{itemize}
\end{frame}
%%%%%%%%%%%%%%%%%%%%%%%%%%%%%%%%%%%%%%%%%%%%%%%%%%%%%%%%
\begin{frame}{Thank you!}
\begin{center}
Thank you for listening!
\end{center}
\end{frame}
\end{document}
